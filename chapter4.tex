\chapter{ روش تحقیق }
\section{مقدمه}\label{sec-model-motion}
در فصل دوم با بررسی آزمایشات انجام شده روی انواع سرریزها مشخص شد که هنوز شکل لبه‌ی سرریزها (اشکال مختلف لبه: ربع دایره، نیم دایره و تیز )خیلی مورد توجه قرار نگرفته و جای خالی سرریزهای ساده و مرکب با لبه‌های گرد حس می‌شود. لازم است رابطه‌ای برای اندازه‌گیری دبی در این نوع سرریزها برای اجرایی و عملی کردن آن‌ها در محیط واقعی ارائه شود.\\
در این فصل ابتدا شرح مختصری در مورد امکانات و ابزار مورد استفاده برای برای مدل‌سازی کانال آورده شده است، در ادامه سرریزهای ساخته شده و نحوه‌ی طراحی آن‌ها آمده و در قسمت پایانی نیز روند انجام آزمایش‌ها در شرایط یکسان هیدرولیکی آورده شده است.
\section{ابزار و امکانات مورد استفاده}
برای انجام آزمایش‌ها از یک فلوم آزمایشگاهی به طول 12 متر، عرض 7/30 سانتی‌متر و عمق نیم متر واقع در آزمایشگاه هیدرولیک دانشکده‌ی کشاورزی بیرجند به عنوان شبیه‌ساز کانال استفاده شده است. این فلوم متشکل از یک پمپ سانتریفیوژ با قدرت 15 اسب بخار، یک کانال با کف فلزی و جداره‌هایی از جنس پلکسی‌گلاس به طول 12 متر، دو مخزن یکی در بالادست و یکی در پایین‌دست فلوم، یک دبی‌سنج الکترونیک، یک اهرم در ابتدای فلوم جهت تنظیم شیب و یک گیج اندازه‌گیری عمق آب با قابلیت حرکت در امتداد طول فلوم است. شکل (\ref{fig25})، پروفیل طولی و پلان این فلوم و شکل (\ref{fig26}) نمای داخلی از فلوم مورد آزمایش در آزمایشگاه را نشان می‌دهد. 
\begin{figure}[h]
\centering
\begin{subfigure}{.5\textwidth}
    \leftline{
  \includegraphics[width=1.5\linewidth]{30}
}
  \caption{ پروفیل طولی فلوم آزمایشگاهی    }
  \label{fig25.1}
\end{subfigure}%
\qquad 
\begin{subfigure}{.5\textwidth}
  \leftline{
  \includegraphics[width=1.5\linewidth]{31}
}
  \caption{  پلان فلوم آزمایشگاهی             }
  \label{fig25.2}
\end{subfigure}
\caption{ }
\label{fig25}
\end{figure}

\begin{figure}[h]
 \centering
  \includegraphics[width=.85\linewidth]{32}
  \caption{    }
  \label{fig26}
\end{figure}

\section{ ساخت مدل سرریز آزمایشگاهی}

جنس مدل سرریزهای ساخته شده ام دی اف بوده ، سرریزهای با لبه‌ی نیم دایره دارای ضخامت 16میلی متر و سرریزهای دیگر دارای ضخامت 8 میلی متر می‌باشند. همانطور که در فصل قبل ذکر شد تاج سرریز لبه تیز مستطیلی باید ضخامتی بین 1 تا 2 میلی‌متر داشته و زاویه شیب لبه‌ی سرریز نیز باید 45 درجه باشد، از اینرو تمام مدل‌های لبه تیز بدین گونه ساخته شده‌اند.\\
در شکل(\ref{fig27}) تعدادی از مدل‌های ساخته شده و در شکل(\ref{fig28}) لبه‌ی تاج آن‌ها آورده شده است.

\begin{figure}[h]
 \rightline{
  \includegraphics[width=.8\linewidth]{33}
}
  \caption{ سرریزهای ساخته شده از جنس ام دی اف    }
  \label{fig27}
\end{figure}

\begin{figure}[h]
\centering
\begin{subfigure}{.5\textwidth}
  \centering
  \includegraphics[width=.8\linewidth]{34}
  \caption{ سرریز مستطیلی با لبه ربع دایره   }
  \label{fig28.1}
\end{subfigure}%
\begin{subfigure}{.5\textwidth}
  \centering
  \includegraphics[width=.8\linewidth]{35}
  \caption{  لبه تیز               }
  \label{fig28.2}
\end{subfigure}
\begin{subfigure}{.5\textwidth}
  \centering
  \includegraphics[width=.8\linewidth]{36}
  \caption{  لبه نیم دایره            }
  \label{fig28.3}
\end{subfigure}
\caption{ }
\label{fig28}
\end{figure}
با توجه به اینکه عرض کانال 30.7 سانتی‌متر و عمق آن 50 سانتی‌متر می‌باشد، ابعاد زیر برای ساخت مدل سرریزهای مرکب مستطیلی- مستطیلی در نظر گرفته شده‌اند که در جدول \ref{table4.1} این اندازه ها آورده شده است.
\begin{table}[h]
\centering
  \caption{  لبه نیم دایره            }  \label{table4.1}
\begin{tabular}{ |c|c| } 
 \hline
   11،16   &    \lr{L}  : طول دهانه مرکزی سرریز \\ 
\hline
 8،13      &     \lr{ P} : ارتفاع تاج سرریز   \\ 
 \hline
\end{tabular}
\end{table}
\section{روش آزمایش}
در این قسمت روند کلی که برای انجام تمام آزمایشات تکرار شده است ، آورده می‌شود. 
\subsection{روند انجام آزمایش}

در هر آزمایش پس از حصول اطمینان از تراز و قائم بودن مدل سرریز ساخته شده در فلوم، مدل توسط چسب شیشه (چسب آکواریوم) به بدنه‌ی کانالِ فلوم نصب و آب‌بندی شده است. پس از نصب مدل سرریز در محل مورد نظر (تقریباً وسط فلوم)، موتور پمپ فلوم را روشن کرده و سپس با گرداندن شیر آب آنرا به مقدار دبی مورد نظر باز کرده و صبر می‌کنیم تا جریان آب با دبی مورد نظر از دهانه‌ی سرریز مرکزی عبور 
کند. دبی توسط دبی سنج الکترونیک بر حسب متر مکعب بر ساعت نشان داده می‌شود. اولین نرخ دبی حدود 4.2 متر مکعب بر ساعت آزمایش شده است. پس از عبور جریان از دهانه‌ی مرکزی سرریز باید مدتی صبر کرد تا جریان یکنواخت گردد. \\
پس از یکنواخت شدن جریان و تشکیل جت آب، عمق یاب را دقیقاً بالای لبه‌ی تاج سرریز تنظیم کرده و نوک عمق یاب را در فاصله‌ی حدود یک میلی متری بالای جریان قرار داده و عدد عمق یاب را یادداشت می کنیم (شکل \ref{fig29}). با حرکت دادن عمق یاب به طرف بالادست جریان عدد روی عمق یاب را یادداشت می کنیم و این کار را تا زمانیکه هد آب ثابت شود ادامه می‌دهیم. (در فاصله‌ی حدود 3-4 برابر $H_d$)
\begin{figure}
 \centering
  \includegraphics[width=.85\linewidth]{37}
  \caption{محل قرار گیری عمق یاب    }
  \label{fig29}
\end{figure}
پس از یادداشت کردن هد جریان در بالادست سرریز با دبی مورد نظر، شیر آب پمپ را کمی بازتر کرده و با میزان دبی جدید مراحل قبل را تکرار می کنیم. لازم بذکر است که با تغییر دبی در هر مرحله حدوداً 10 دقیقه طول می‌کشد تا جریان عبوری از سرریز یکنواخت گردد.
این روند را برای6 دبی مختلف برای سرریز ساده‌ی مرکزی انجام داده تا به آستانه‌ی سرریز مرکب برسیم. سپس مثل حالت قبل با افزایش دبی، آزمایشات مربوط به سرریز مرکب را نیز برای 5 دبی مختلف تکرار می کنیم.
برای بررسی تأثیر استغراق بر ضریب دبی، آزمایشات لازم بصورت زیر انجام شده است :
روند کلی انجام آزمایشات همانند حالت جریان آزاد است با این تفاوت که هر بار بعد از روشن کردن موتور پمپ و باز کردن شیر آب تا دبی مورد نظر، شروع به باز کردن دریچه‌ی انتهایی فلوم می‌کنیم. باز کردن دریچه را تا زمانیکه سطح پایاب به بالادست جریان اثر گذاشته و باعث بالا آمدن عمق آب بالادست سرریز گردد ادامه می‌دهیم. سپس عمق‌یاب را دقیقاً روی تاج سرریز قرار داده و عدد عمق‌یاب را یادداشت می‌کنیم ($H_d$). عمق‌یاب را در جهت بالادست جریان تا زمانی حرکت می‌دهیم که عمق‌یاب عدد ثابتی را نشان دهد(در فاصله‌ی حدود 3-4 برابر $H_d$). همین کار را برای پایین دست سرریز نیز تکرار کرده و عمق‌یاب را تا زمانیکه عدد ثابتی را نشان دهد (در فاصله‌ی حدود 6-8 برابر $H_d$) در جهت پایین‌دست سرریز  حرکت می‌دهیم و عدد عمق‌یاب را یادداشت می‌کنیم ($H_1$). سپس دبی جریان را تغییر می‌دهیم و آزمایشات را برای 5 دبی مختلف برای سرریز ساده و برای 4 دبی مختلف برای سرریز مرکب تکرار می‌کنیم.
\subsection{ترتیب انجام آزمایشات}
در جدول زیر ترتیب انجام آزمایشات آمده است. برای هر مدل 11دبی مختلف مورد آزمایش قرار گرفته است.\\
\begin{table}[h]
\caption {ترتیب انجام آزمایشات } \label{table4.2} 
\centering
\begin{tabular}{ | c | c | P{3cm} | P{3cm} | }
\hline
\rowcolor{Gray1}
ترتیب آزمایشات      &	شکل لبه‌ی سرریز&	  طول دهانه‌ی مرکزی سرریز،\lr{L(cm)}  &	ارتفاع قسمت مرکزی سرریز از کف،\lr{p(cm)} \\
\hline\rowcolor{Gray}
1     &	نیم دایره     &	11  &	8\\  \cline{1-4}
\rowcolor{Gray}
2     &	 نیم دایره   &	11    &	13\\
\hline\rowcolor{Gray}
3     &	نیم دایره     &	16   &	8\\
\hline\rowcolor{Gray}
4    &	نیم دایره    &	16  &	13\\
\hline\rowcolor{Gray1}
  5   &	 ربع دایره    &	11  &	8\\
\hline\rowcolor{Gray1}
   6  &	 ربع دایره   &	11	& 13\\
\hline\rowcolor{Gray1}
7   &	ربع دایره    &	16  &	8\\
\hline\rowcolor{Gray1}
8   &	ربع دایره    &	16   &	13\\
\hline\rowcolor{Gray}
9    &	لبه تیز &	11  &	8\\
\hline\rowcolor{Gray}
10  &	لبه تیز   &	  11  &	13\\
\hline\rowcolor{Gray}
11  &	لبه تیز  &	16  &	8\\
\hline\rowcolor{Gray}
12  &	لبه تیز  &	16 &	13\\
\hline
 \end{tabular}
\end{table}
\newline
همانطور که در جدول فوق نشان داده شده است، برای هر شکل لبه چهار آزمایش با دو نسبت فشردگی جانبی و دو ارتفاع تاج انجام گرفته است. آزمایشات فوق برای جریان مستغرق نیز تکرار شده است.










