\chapter{نتیجه‌گیری و پیشنهادات}
در پروژه‌ی جاری به بررسی آزمایشگاهی سرریزهای مرکب مستطیلی- مستطیلی با هدف مطالعه‌ی تأثیر پارامترهای هندسی ( ارتفاع تاج سرریزو طول دهانه‌ی سرریز)، شکل لبه‌ی تاج سرریز و استغراق جریان روی ضریب دبی پرداخته شد. نتایج این پروژه و هم‌چنین پیشنهادات برای کارهای آینده در ذیل آمده است.
\section{نتیجه‌گیری}
با مرور مطالب ذکر شده در فصل پنجم خلاصه‌ی نتایج در ادامه ارائه می‌گردد.
\subsection{سرریزهای ساده‌ی مستطیلی}
دبی عبوری از روی سرریزهای ساده‌ی مستطیلی تابع پارامترهای ارتفاع تاج سرریز($P$)، ارتفاع آب بالادست سرریز نسبت به تاج سرریز($H_d$)، ابعاد دهانه‌ی سرریز($L$)، عرض کانال($B$) و شتاب ثقل می‌باشد. به منظور بررسی اثر پارامترهای فوق روی عملکرد سرریز لازم است در یک تحقیق تمام این پارامترها تغییر کنند. در این پروژه از میان پارامترهای فوق قادر به تغییر عرض کانال و شتاب ثقل نبوده ولی سایر پارامترها تغییر یافته‌اند.\\
اثر پارامترهای هد آب بالادست سرریز ($H_d$)، ارتفاع تاج سرریز($P$)، ابعاد اضلاع دهانه‌ی سرریز($L$)، شکل لبه‌ی تاج سرریز و درصد استغراق روی ضریب دبی مطالعه شد و نتایج حاصل بطور مختصر عبارتند از:

\begin{enumerate}
\item		ارتفاع آب بالادست سرریز و ضریب دبی با هم رابطه‌ی عکس دارند. یعنی با افزایش ارتفاع آب روی سرریز ضریب دبی سرریزهای ساده‌ی مستطیلی(لبه تیز، لبه نیم دایره و لبه ربع دایره) کاهش می‌یابد و بالعکس.
\item	 	در یک سرریز ساده‌ی مستطیلی لبه‌تیز، در یک مقدار مشخص      $\frac{H_d}{P}$        با افزایش ارتفاع تاج سرریز ضریب دبی کاهش می‌یابد ولی در سرریزهای ساده‌ی مستطیلی با لبه‌ی ربع دایره و لبه نیم دایره در یک    $\frac{H_d}{P}$         ثابت با افزایش ارتفاع تاج سرریز ضریب دبی نیز افزایش می‌یابد.

\item در سرریزهای ساده‌ی مستطیلی لبه‌تیز و لبه‌‌ربع دایره هنگامیکه    $\frac{H_d}{P}$         مقداری ثابت و مشخص باشد، با افزایش طول دهانه‌ی مرکزی سرریز ضریب دبی نیز افزایش می‌یابد. اما در سرریزهای ساده‌ی مستطیلی با لبه‌ی نیم دایره در یک مقدار مشخص     $\frac{H_d}{P}$           با افزایش طول سرریز ضریب دبی کاهش می‌یابد.
\item		با ثابت بودن طول دهانه‌ی سرریز و در یک مقدار مشخص         $\frac{H_d}{P}$       ، سرریزهای ساده‌ی مستطیلی لبه‌تیز کم‌ترین مقدار ضریب دبی و سرریزهای ساده‌ی مستطیلی با لبه‌ی نیم‌دایره بیشترین ضریب دبی را دارا می‌باشند.
\item	در سرریزهای ساده‌ی مستطیلی (لبه‌تیز، لبه نیم دایره و لبه ربع دایره) با افزایش میزان استغراق، ضریب دبی جریان کمتر می‌شود. 
\end{enumerate}
بعد از بررسی اثرات پارامترهای فوق روی ضریب دبی سرریزهای ساده‌ی مستطیلی، درصد خطاهای دبی آزمایشگاهی و دبی محاسباتی با روش‌های مختلف برای سرریزهای ساده‌ی مستطیلی لبه‌تیز مقایسه می‌شود:

\begin{enumerate}
\item  
	بعد از مقایسه‌ی دبی اندازه‌گیری شده و دبی محاسباتی برای سرریزهای ساده‌ی مستطیلی لبه‌تیز با ارتفاع تاج یکسان و طول دهانه‌ی مرکزی متفاوت، روش هندرسن کم‌ترین درصد خطا ($5.93$) و روش سوئیس بیشترین درصد خطا ($14.53$) را دارد.
\item 
	بعد از مقایسه‌ی دبی اندازه‌گیری شده و دبی محاسباتی برای سرریزهای ساده‌ی مستطیلی لبه‌تیز با طول دهانه‌ی مرکزی یکسان و ارتفاع تاج  متفاوت، روش هندرسن کم‌ترین درصد خطا ($7.077$) و روش سوئیس بیشترین درصد خطا ($15.85$) را دارد.
\end{enumerate}
\subsection{سرریزهای مرکب مستطیلی- مستطیلی}
دبی عبوری از روی سرریزهای مرکب مستطیلی- مستطیلی تابع پارامترهای ارتفاع تاج سرریز ($P_1,P$)، ارتفاع آب بالادست سرریز نسبت به تاج سرریز ($H_d$)، ابعاد دهانه‌ی سرریز ($L_1,L$)،  عرض کانال ($B$) و شتاب ثقل ($g$) می‌باشد. به منظور بررسی اثر پارامترهای فوق روی عملکرد سرریز لازم است در یک تحقیق تمام این پارامترها تغییر کنند. در این پروژه از میان پارامترهای فوق قادر به تغییر عرض کانال و شتاب ثقل نبوده ولی سایر پارامترها تغییر یافته‌اند.\\
اثر پارامترهای هد آب بالادست سرریز ($H_d$)، ارتفاع تاج سرریز ($P_1,P$)،   ابعاد اضلاع دهانه‌ی سرریز ($L_1,L$)، شکل لبه‌ی تاج سرریز و درصد استغراق روی ضریب دبی مطالعه شد و نتایج حاصل بطور مختصر عبارتند از:

\begin{enumerate}
\item  
		ارتفاع آب روی سرریز و ضریب دبی سرریزهای مرکب مستطیلی- مستطیلی  با هم رابطه‌ی عکس دارند، به گونه‌ای که با افزایش عمق آب ضریب دبی کاهش می‌یابد و بالعکس.
\item 
	در سرریزهای مرکب مستطیلی- مستطیلی(لبه‌تیز، لبه‌ربع‌دایره و لبه‌نیم‌دایره)، در یک مقدار مشخص     $\frac{H_d}{P}$    با افزایش ارتفاع تاج سرریز ضریب دبی کاهش می‌یابد یعنی ارتفاع تاج سرریز و ضریب دبی در سرریزهای مرکب مستطیلی- مستطیلی با هم رابطه‌ی عکس دارند.
\item
	در سرریزهای مرکب مستطیلی- مستطیلی لبه‌تیز هنگامیکه         $\frac{H_d}{P}$     مقداری ثابت و مشخص باشد، با افزایش طول دهانه‌ی مرکزی سرریز ضریب دبی نیز افزایش می‌یابد. اما در سرریزهای مرکب مستطیلی- مستطیلی با لبه‌ی نیم دایره و لبه‌ی ربع دایره در یک مقدار مشخص  $\frac{H_d}{P}$ با افزایش طول سرریز ضریب دبی کاهش می‌یابد.
\item

	با ثابت بودن طول دهانه‌ی سرریز و در یک مقدار مشخص       $\frac{H_d}{P}$     ، سرریزهای مرکب مستطیلی- مستطیلی لبه‌تیز کم‌ترین مقدار ضریب دبی و سرریزهای مرکب مستطیلی- مستطیلی با لبه‌ی نیم‌دایره بیشترین ضریب دبی را دارا می‌باشند.

\item
در سرریزهای مرکب مستطیلی- مستطیلی (لبه‌تیز، لبه نیم دایره و لبه ربع دایره) با افزایش میزان استغراق، ضریب دبی جریان کمتر می‌شود.
\end{enumerate}
بعد از بررسی اثرات پارامترهای فوق روی ضریب دبی سرریزهای مرکب مستطیلی- مستطیلی، درصد خطاهای دبی آزمایشگاهی و دبی محاسباتی با روش‌های مختلف برای سرریزهای مرکب مستطیلی لبه‌تیز مقایسه می‌شود:
\begin{enumerate}
\item
در مقایسه‌ی دبی محاسباتی و دبی اندازه‌گیری شده برای سرریزهای مرکب مستطیلی- مستطیلی لبه‌تیز با ارتفاع تاج یکسان و طول دهانه‌ی متغیر، روش کیندزواتر کم‌ترین درصد خطا($4.63$) و روش هندرسن بیشترین درصد خطا($12.7$) را دارد.
\item
	در مقایسه‌ی دبی محاسباتی و دبی اندازه‌گیری شده برای سرریزهای مرکب مستطیلی- مستطیلی لبه‌تیز با طول دهانه‌ی یکسان و ارتفاع تاج متغیر، روش کیندزواتر کم‌ترین درصد خطا($6.34$) و روش هندرسن بیشترین درصد خطا($9.63$) را دارد.

\end{enumerate}

\section{پیشنهادات}
با توجه به محدودیت‌های زمانی و امکانات آزمایشی، پیشنهادات زیر جهت کمک به مطالعات آتی توصیه می‌گردد:

\begin{enumerate}
\item	از آنجا که در این پروژه سرریزهای مرکب مستطیلی مورد بررسی قرار گرفتند توصیه می‌شود که این آزمایش‌ها برای سرریزهای دیگر نیز انجام شده، تأثیر شکل لبه‌ی تاج بر عملکرد آن‌ها مورد بررسی قرار گرفته و نتایج حاصل با نتایج این پروژه مقایسه شود.
\item
	آزمایش‌های این پروژه در فلومی به عرض $7/30$ سانتی‌متر انجام شده است. توصیه می‌شود که این آزمایش‌ها در فلوم‌های دیگر با عرض‌های متفاوت نیز صورت گرفته و نتایج حاصل با فرمول‌ها و نتایج این پروژه مقایسه شود.
\item 
	از آنجا که در این پروژه آزمایش‌ها بر روی مدل‌هایی با دو نسبت فشردگی جانبی($L/B$) و دو ارتفاع تاج مورد بررسی قرار گرفته، توصیه می‌شود که آزمایش‌ها را بر روی تعداد بیشتری مدل با نسبت‌ فشردگی و ارتفاع تاج جدید توسیع داد و نتایج حاصل را با نتایج این پروژه مقایسه کرد.
\item
	توصیه می‌شود که با نرم افزارهای \lr{Fluent} و \lr{ Flow3D}  مدل‌های این پروژه شبیه‌سازی شود و نتایج حاصل از شبیه‌سازی عددی با نتایج آزمایشگاهی این پروژه مقایسه گردد.

\item
	در این پژوهش به دلیل محدودیت حرکت ارتفاع سنج، فقط نقاط راستای طولی کانال برداشت شده‌‌اند، لذا توصیه می‌شود که نقاط راستای عرضی کانال در بالادست سرریز نیز برداشت شوند و مورد بررسی قرار گیرند.
\end{enumerate}






















