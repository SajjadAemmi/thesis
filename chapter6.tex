\chapter{نتیجه‌گیری و پیشنهادات}

\section{مقدمه}
هدف کار ارائه شده در این پژوهش ارائه و بهبود روش‌های خودکار مبتنی بر هوش مصنوعی و یادگیری ژرف به منظور کمک به جامعه پزشکی در تشخیص زودهنگام رتینوپاتی دیابتی است و همچنان راه حل‌ها و ایده‌های دیگری می‌تواند در جهت پیشرفت تشخیص بیماری از جمله رتینوپاتی دیابتی کمک کننده باشند.
\section{بحث و نتیجه‌گیری}
روش‌های مختلفی برای شناسایی زودهنگام رتینوپاتی دیابتی و شناسایی میکروآنوریسم ارائه شده‌اند. دسته‌ای از این روش‌ها سعی کرده‌اند تا به کمک روش‌های غیر ژرف و استخراج ویژگی‌های معنادار بین پیکسل‌های میکروآنوریسم‌ و غیرمیکروآنوریسم‌ تفاوت قائل شوند و در نهایت یک دسته‌بند را با ویژگی‌های ارائه شده آموزش دهند و از این دسته‌بند برای آزمون استفاده کنند.
\noindent
در مقابل دسته‌ای دیگر از روش‌ها وجود دارند که با تکیه بر ویژگی‌های ژرف استخراج شده از شبکه‌های عصبی پیچشی سعی دارند پیکسل‌های میکروآنوریسم را شناسایی کنند. به این صورت که تصاویر آموزشی را به شبکه می‌دهند و شبکه در مرحله آموزش سعی می‌کند ویژگی‌های لازم و تفکیک‌پذیر بین میکروآنوریسم‌ها و غیرمیکروآنوریسم‌ها را شناسایی کند. واضح است که ویژگی‌های استخراج شده توسط شبکه با ویژگی‌های معنادار استخراج شده توسط انسان همیشه برابر نیستند و حتی ممکن است ویژگی‌های شبکه اصلا قابل تفسیر نباشند.
\noindent
پس در این جا با دو دسته از ویژگی‌ها رو به رو هستیم. در این پژوهش سعی شده است که از هر دو دسته ویژگی استفاده شود. بنابراین ما ویژگی‌های معنادار مطلوب را در کنار ویژگی‌های استخراج شده توسط شبکه قرار داده‌ایم تا میزان دقت و حساسیت شبکه را بالا ببریم.
\section{پیشنهادات}
به نظر می‌رسد استفاده از ویژگی‌های معنادار در کنار ویژگی‌های ژرف در تشخیص میکروآنوریسم بسیار بهتر عمل می‌کند. به منظور بهبود در نتایج باید در انتخاب ویژگی‌ها از هر دو دسته دقت بیشتری داشت.
بنابراین برای بهبود ویژگی‌های ژرف، استفاده از ویژگی‌های چند مدل شبکه در کنار هم می‌تواند کمک کننده باشد. به این صورت که هر مدل ویژگی‌های مختلف را استخراج می‌کند و از کنار هم قرار دادن چند دسته ویژگی ژرف، بردار ویژگی ما غنی‌تر خواهد شد. 
\noindent
همچنین می‌توان بهترین اندازه تکه را برای هر یک از مدل‌ها به دست آورد و به این ترتیب یک شبکه ترکیبی چند مقیاسی به دست خواهد آمد که از چند مدل‌ قدرتمند و با اندازه تکه‌های مختلف برای تشخیص میکروآنوریسم استفاده می‌کند. 
\noindent
از طرفی در این میان ممکن است ویژگی‌هایی یکسان از شبکه‌ها استخراج شود. و یا این که ویژگی‌هایی وجود داشته باشند که در تشخیص میکروآنوریسم زیاد نقش موثری نداشته باشند. به این ترتیب استفاده از روش‌های کاهش بردار ویژگی برای استخراج ویژگی‌های اصلی برای ایجاد مرز بین پیکسل‌های میکروآنوریسم و غیر میکروآنوریسم می‌توان مفید باشد.
در مقابل با تحقیق می‌توان ویژگی‌های متمایز کننده معنادار دیگری برای پیکسل‌های میکروآنوریسم به دست آورد تا به بردار ویژگی اضافه شوند.
روش‌های مختلفی برای پیش‌پردازش تصاویر فوندوس وجود دارند. استفاده از روش‌های جدید پیش‌پردازش به منظور کاهش بیشتر نویز، یکسان‌ سازی روشنایی تصاویر بافت‌نگار و ... می‌تواند به استخراج ویژگی‌های مفیدتر کمک کند تا بتوان نتایج بهتر برای تشخیص زودهنگام رتینوپاتی دیابتی و میکروآنوریسم به دست آورد.
