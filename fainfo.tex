% در صورت تمایل به تغییر لوگوی بسم‌ الله، لوگوی دلخواه خود را با اسم logo در پوشه figures قرار دهید و سپس با استفاده از دستور
% زیر پهنای آن را با عددی بین 0 و 1 مشخص کنید
%\besmwidth{.7}
% دانشکده، آموزشکده و یا پژوهشکده  خود را وارد کنید
\faculty{دانشکدۀ مهندسی }
% گروه آموزشی خود را وارد کنید
\department{گروه مهندسی کامپیوتر}
% نام رشته تحصیلی خود را وارد کنید
\subject{مهندسی کامپیوتر}
% گرایش خود را وارد کنید
\field{هوش مصنوعی}
% عنوان پایان‌نامه/رساله را وارد کنید
\title{تشخیص بی‌درنگ چهره در محیط های بدون محدودیت}
% نام پژوهشگر را وارد کنید
\name{سید سجاد	}
% نام خانوادگی پژوهشگر را وارد کنید
\surname{اعمی}
% نام استاد(ان) راهنما را وارد کنید
\firstsupervisor{دکتر حمید رضا پور رضا}
%\secondsupervisor{}
% نام استاد(دان) مشاور را وارد کنید. چنانچه استاد مشاور ندارید، دستور(های) پایین را غیرفعال کنید.
\firstadvisor{دکتر بشرا رجائی}
%\secondadvisor{استاد مشاور دوم}
% تاریخ پایان‌نامه را وارد کنید
\thesisdate{شهریور‌ماه ۱۴۰۰ }
\credit{۶}
\defensedate{۱۴۰۰/۰۶/۰۶}
% در صورت کامنت کردن سه دستور زیر، جای خالی به جای آن‌ها در فرم صورت‌جلسه ایجاد می‌شود
%\grade{۲۰}
%\letgrade{بیست}
%\degree{عالی}
% داور یا صاحبنظر داخلی اول
\firstinternalreferee{دکتر    }
% داور یا صاحبنظر داخلی دوم
%\secondinternalreferee{دکتر رضا}
% داور یا صاحبنظر خارجی اول
\firstexternalreferee{}
% داور یا صاحبنظر خارجی دوم
%\secondexternalreferee{دکتر بیژن}
% ناظر جلسه دفاع
%\viewer{دکتر احمد}
%%%%%%%%%%%%%%%%%%%%%%%%%%%%%%%%%%%%
\totext{%
\noindent{\Large\bfseries تقدیم به }\\
\vspace*{1em}
\begin{center}
\large\bfseries
پدر و مادر عزیزم
\end{center}
\begin{center}
و همه کسانی که درست اندیشیدن را به من آموختند.
%%%%دکتر علی شریعتی
\end{center}
}
%%%%%%%%%%%%%%%%%%%%%%%%%%%%%%%%%%%%
\ack{%
\subsection*{سپاس‌گزاری}
سپاس خداوند یکتای عزتمندی که رحمت و دانش او در سراسر گیتی گسترده شده، آسمان‌ها و زمین همه از آن اوست و  علم و دانش حقیقی را بر هر که بخواهد موهبت می‌فرماید. رحمت و لطف او را بی‌نهایت سپاس می‌گویم چرا که فهم و درک مطالب این پژوهش را بر من ارزانی داشت و مرا به این اصل رساند که علم و ایمان دو بال یک پروازند. توفیق تلاش به‌ من داد و هر بار که خطا کردم فرصتی دوباره، تا با امید، تلاشی تازه را آغاز کنم و به خواست او به نتیجه‌ی مطلوب نائل آیم. به‌راستی که همه چیز از آن اوست و همه‌چیز به خواست اوست. 
%%%% در صورت استفاده از دستور زیر، تاریخ و امضای شما، به طور خودکار درج می‌شود.
%\signature 
}
%%%%%%%%%%%%%%%%%%%%%%%%%%%%%%%%%%%%
\faabstract{%
\subsection*{چکیده}

در سال های اخیر، به دلیل استفاده از یادگیری عمیق، فناوری تشخیص چهره شاهد پیشرفت‌های چشمگیری بوده است. با این حال، استراتژی‌های داده محور یک چالش را به همراه می‌آورد: تصاویر ارسال شده به سامانه تشخیص چهره همیشه برای تشخیص مناسب نیستند و ممکن است چهره هایی با وضوح کم، چهره های تار در حال حرکت، صورت های مسدود و حتی مشکلاتی در پس زمینه وجود داشته باشد. متأسفانه، از آنجا که موتور تشخیص چهره قبلاً چنین چهره‌های بی کیفیتی را ندیده است، احتمالاً تصمیمات نادرستی در مورد آن‌ها می‌گیرد.

\noindent
این پایان نامه با محوریت موضوع تشخیص بی‌درنگ چهره در محیط‌های کنترل نشده می‌باشد که از دو بخش اصلی یافتن چهره و شناسایی چهره تشکیل شده است. روش پیشنهادی در بخش شناسایی چهره می‌باشد. هدف نهایی طراحی بخش نرم افزاری یک عینک هوشمند برای افراد نابینا می‌باشد. هنگامی که فرد نابینا از عینک استفاده کرده و در محیط‌های عمومی به راه رفتن بپردازد، دوربینی که روی عینک نصب شده است، چهره افرادی که در زاویه دید آن قرار دارند را بررسی کرده و در صورت یافتن یک چهره آشنا، فرد مورد نظر شناسایی شده و نامش از طریق صدا برای فرد نابینا خوانده می‌شود.

\noindent
در پیاده سازی این سامانه که باید در مکان‌های عمومی، معابر پیاده و محیط‌های کنترل نشده مورد استفاده قرار بگیرد، چند چالش مهم مانند نورپردازی غیر یکنواخت، انسداد، تاری خارج از تمرکز دوربین و زاویه نا‌مطلوب چهره نسبت به دوربین وجود دارد. از طرفی این سامانه باید به صورت بی‌درنگ رفتار نماید. زیرا فرصت زیادی برای شناسایی فردی که در خیابان در کنار دوربین می‌گذرد، وجود ندارد. از سوی دیگر به دلیل اجرای پردازش‌ها بر روی پردازشگر تلفن همراه، باید محدودیت‌ منابع را نیز در نظر گرفت و الگوریتم استفاده شده باید دارای کمترین پیچیدگی زمانی و حافظه باشد. بدین منظور مبنای تحقیق را بر معماری MobileNetV3 قرار دادیم. در این پایان نامه محوریت اصلی مطالعات بر روی طراحی یک الگوریتم کارا و مناسب برای مرحله شناسایی بی‌درنگ‌ چهره در شرایط بدون محدودیت است.

\noindent
علاوه بر موارد گفته شده در بالا، ما فرض کردیم که داده‌های محدودی از هر دسته در اختیار داریم. برای مقابله با این چالش از روش‌های  \lr{one shot learning} و  \lr{few shot learning} استفاده می‌نماییم‌. با بررسی نتایج حاصل از این پژوهش بر روی تصاویر مجموعه داده \lr{LFW}  و \lr{YouTube Faces}، دقت روش پیشنهادی ما به ترتیب برابر با ۹۶ \% و ۹۴ \% شد که دقت بالاتری نسبت به روش‌های مشابه می‌باشد. 

}
\yazdtitle
