\chapter{ارزیابی روش پیشنهادی}
\section{مقدمه}


مقدمه
در این کار سعی بر این داشته‌ایم تا به کمک روش‌های یادگیری ژرف در راستای شناسایی میکروآنوریسم‌ها در تصاویر فوندوس قدم برداریم و در حوزه پزشکی عملکرد هوش مصنوعی و یادگیری ژرف را بهبود دهیم. بنابراین با تحقیق و آزمایش روشی برای شناسایی میکروآنوریسم در تصاویر فوندوس پیشنهاد داده‌ایم که شرح آن در فصل سوم انجام شد و نوبت آن است که الگوریتم پیشنهادی را با نمونه‌های مشابه مقایسه کنیم.
پیاده‌سازی این الگوریتم به کمک زبان برنامه‌ نویسی پایتون و کتابخانه Keras انجام شده است.[35]. محیط برنامه نویسی این کار بر روی سرویس ابری Google Colab می‌باشد که حدود 25 گیگابایت حافظه و حدود 70 گیگابایت فضای دیسک در اختیار گرفته‌ایم.
از کتابخانه‌های مهم مورد استفاده در این کار می‌‌توان به کتابخانه Keras و Tensorflow [36] با هدف آموزش و ارزیابی مدل‌های یادگیری ژرف، OpenCV [37] به منظور پردازش تصاویر، NumPy [38] برای انجام محاسبات ریاضی و ماتریسی و کار با آرایه‌ها اشاره کرد.
پس از بیان معیارهای ارزیابی، به بیان نتایج و مقایسه با کارهای دیگر می‌پردازیم.
معیارهای ارزیابی
مفاهیم اولیه
قبل از پرداختن به معیارهای ارزیابی نمادها و مفاهیم اولیه را بیان می‌کنیم:
	مثبت‌های صحیح  (TP): نمونه‌های مثبت (بیمار) که به درستی بیمار تشخیص داده شده‌اند.
	مثبت‌های کاذب  (FP): نمونه‌های منفی (سالم) که به اشتباه بیمار تشخیص داده شده‌اند.
	منفی‌های صحیح  (TN): نمونه‌های منفی (سالم) که به درستی سالم تشخیص داده شده‌اند.
	منفی‌های کاذب  (FN): نمونه‌های مثبت (بیمار) که به اشتباه سالم تشخیص داده شده‌اند.
دقت
یکی از معیارهایی که بسیار در پزشکی و سایر زمینه‌های تحقیقاتی حائز اهمیت است، معیار دقت می‌باشد. در این کار ما با تصاویر دارای میکروآنوریسم و تصاویر سالم سر و کار داریم. بنابراین معیار دقت در این کار به معنی درصد نمونه‌هایی است که بیمار و یا سالم بودن آن‌ها به درستی تشخیص داده شده است. فرمول این معیار در رابطه (4-1) بیان شده است.
\begin{equation}\label{eq3-1}
Accuracy=\ \frac{TP+TN}{N}
\end{equation}
که N تعداد کل نمونه‌ها را نشان می‌دهد.
حساسیت 
در حوزه پزشکی و تشخیص ضایعه‌ها، عارضه‌ها و بیماری‌ها، معیار حساسیت بسیار اهمیت ویژه و بالاتری نسبت به دقت دارد. به این معنی که هر چقدر درصد شناسایی عارضه یا بیماری از بین نمونه‌های آزمون بیمار بیشتر باشد الگوریتم و روش ما ارزشمندتر خواهد بود. حساسیت نیز بیانگر همین موضوع است و درصد تشخیص صحیح نمونه‌های بیمار را نشان می‌دهد. 
دلیل ارزشمند بودن این معیار این است که نمونه‌های بیمار باید حتما به درستی تشخیص داده شوند و اگر نمونه بیماری به اشتباه سالم تشخیص داده شود می‌تواند تبعات جبران ناپذیری برای بیمار به همراه داشته باشد. رابطه      (4-2) رابطه حساسیت را نشان می‌دهد.


ویژگی 
در مقابل معیار حساسیت، معیار تحت عنوان ویژگی وجود دارد. این ویژگی درصد نمونه‌های سالمی که به درستی سالم تشخیص داده شده‌اند را نشان می‌دهد. فرمول این معیار را در رابطه (4-3) عنوان کرده‌ایم.

F-Score
با توجه به سه معیاری که ذکر شد مشخص می‌شود که معیار حساسیت از اهمیت بیشتری نسبت به دو معیار دیگر برخوردار است. اما اگر سیستمی را در نظر بگیریم که تمام نمونه‌ها را بیمار تشخیص دهد، این سیستم حساسیت 100% را داراست اما ویژگی این سیستم صفر درصد خواهد بود. زیر هیچ نمونه سالمی را به درستی تشخیص نداده است. بنابراین معیار حساسیت به تنهایی نمی‌تواند معیاری ارزیابی یک سیستم تشخیص بیماری (در کار ما تشخیص میکروآنوریسم) باشد. زیرا معیار ویژگی نیز اهمیت خود را نشان می‌دهد.
به همین منظور به دنبال معیاری خواهیم بود که معیارهای حساسیت و ویژگی را به هم در کنار یکدیگر داشته باشد. برای این کار مقدار یکی از عامل‌ها، نظیر آستانه تشخیص بیماری را آن قدر عوض می‌کنیم تا میانگین تعداد مثبت‌های کاذب در هر تصویر آزمون به مقادیر مشخص یک، دو، چهار و هشت برسد. حال در این مقادیر مشخص میزان حساسیت به دست آمده را محاسبه می‌کنیم. سپس به کمک تخمین، میزان حساسیت را به ازای مقادیر انتزاعی یک دوم، یک چهارم و یک هشتم محاسبه می‌کنیم و بین حساسیت‌های به دست آمده میانگین می‌گیریم. مقدار به دست آمده معیار F-Score را مشخص می‌کند که میانگین حساسیت را به ازای مقادیر مشخص مثبت‌های کاذب در هر تصویر نشان می‌دهد.
این معیار مثبت‌های کاذب را در خود جای داده است و حساسیت را در شرایط خاص یکسان بررسی می‌کند. فرمول محسابه این معیار را به صورت رابطه (4-4) ذکر کرده‌ایم.

که در این رابطه بیان کننده حساسیت در حالتی است که تعداد مثبت‌های کاذب برابر با مقدار fp باشد. N تعداد کل آستانه‌هایی است که به ازای آن مقدار حساسیت را محاسبه کرده‌ایم.
مجموعه داده
در این کار از دو مجموعه داده برای آموزش و آزمون استفاده شده است که به شرح آن‌ها می‌پردازیم.
مجموعه داده E-Ophtha
این بانک شامل دو مجموعه داده زیر مجموعه با نام‌های E-ophtha-MA برای میکروآنوریسم‌ها و همچنین مجموعه داده E-ophtha-EX برای اگزودا است. مجموعه داده میکروآنوریسم شامل 148 تصویر دارای میکروآنوریسم و هموریج‌های ریز و 233 تصویر بدون عارضه است.[39].
مجموعه داده DIARETDB1
این مجموعه داده شامل 89 تصویر است که به صورت دستی در پنج دسته قرار گرفته‌اند که این پنج دسته معرف حالت‌ها و درجه‌های مختلف رتینوپاتی دیابتی است. 27 تصویر سالم، 7 تصویر با رتینوپاتی دیابتی خفیف، 
28 تصویر با رتینوپاتی دیابتی متوسط و شدید بدون نورگ‌زایی و در نهایت 27 تصویر رتینوپاتی دیابتی شدید به همراه نورگ‌‌زایی است. در کل، مجموعه داده به دو دسته تصاویر آموزش شامل 28 تصویر و تصاویر آزمون شامل 61 تصویر تقسیم شده‌اند. در برچسب زدن به هر تصویر چهار خبره نقش داشته‌اند.[40]. در این کار ما مناطقی که دارای بیش از 75 درصد نظر موافق برای میکروآنوریسم وجود دارد را به عنوان مناطق اصلی میکروآنوریسم در نظر می‌گیریم.

پیکربندی الگوریتم
به منظور آموزش شبکه‌ها، در هر دوره 20% تعداد داد‌ه‌های آموزشی را به عناون داده‌های ارزیابی در نظر می‌گیریم تا روند آموزش شبکه را بر اساس عامل‌های دیگر مورد بررسی قرار دهیم. همچنین در هر تکرار، تعداد دسته‌هایی که به شبکه برای آموزش داده می‌شود را برابر 16 قرار دادیم و از بهینه‌ساز Adam جهت آموزش استفاده کرده‌ایم. این بهینه‌ساز نیز نرخ آموزش را بر اساس خطا به صورت تطبیقی کم یا زیاد می‌کند. همچنین میانگین کاهش گرادیان‌های تکرارهای قبل را نگهداری می‌کند تا بر اساس آن‌ها جهت گرادیان تکرار جدید را محاسبه کند. 
برای تابع ضرر از تابع آنتروپی تقاطعی دسته‌ای  استفاده کرده‌‌ایم. این تابع میزان ضرر را به صورت رابطه (4-5) محاسبه می‌کند.

که در این تابع N اندازه خروجی شبکه (در کار ما N برابر با دو است) را نشان می‌دهد.ین مقدار از برچسب اصلی  iامین مقدار از بردار خروجی پیش‌بینی شده توسط شبکه را نشان می‌دهد. 
همچنین برای به دست آوردن بهترین نتیجه از آموزش، از روش ارزیابی تقاطعی استفاده کرده‌ایم. به این صورت که در هر مرتبه آموزش، تعداد داده‌های آموزش را به پنج دسته تقسیم می‌کنیم. چهار قسمت را برای آموزش و یک قسمت را برای آزمون در نظر می‌گیریم. به این ترتیب پنج مدل شبکه برای آموزش خواهیم داشت و بهترین نتیجه را برای آزمون بر روی داده‌های آزمون انتخاب می‌کنیم.
نتایج آزمون 
با توجه به پیکربندی بیان شده و همچنین عامل‌هایی که در فصل سوم توضیح داده شد، مدل را آموزش داده‌ایم و به سراغ آزمون داده‌های آزمون می‌رویم. آستانه شبکه عصبی پیچشی دوم که ویژگی‌های مبتنی بر پیکسل را برای دسته‌بندی نهایی استخراج می‌کند آن قدر تغییر داده‌ایم تا میانگین تعداد مثبت‌های کاذب در هر تصویر آزمون برابر با مقادیر یک، دو، چهار و هشت شود.
حال در هر یک از چهار آستانه به دست آمده، به طور مجزا میانگین حساسیت بر روی تصاویر آزمون به دست آورده‌ایم و سپس با تخمین، میانگین حساسیت را در شرایطی که تعداد مثبت‌های کاذب برابر با یک هشتم، یک چهارم و یک دوم باشد حساب کرده‌ایم. در جدول ‏4 1 و جدول ‏4 2 نتایج حاصل از محاسبه حساسیت را در مقادیر مختلف مثبت‌های کاذب در دو مجموعه داده E-Ophtha-MA و DIARETDB1 مشاهده می‌کنید.



نمودار FROC که مقدار حساسیت را به ازای مقادیر مختلف مثبت‌های کاذب نشان می‌دهد، برای مجموعه داده‌های E-Ophtha-MA و DIAREDDB1 در شکل ‏4 1 و شکل ‏4 2 مشاهده می‌کنید.

 
شکل ‏4 1- منحنی FROC مقالات در مقایسه با روش پیشنهادی برای مجموعه داده E-Ophtha-MA




 
شکل ‏4 2- منحنی FROC مقالات در مقایسه با روش پیشنهادی برای مجموعه داده DIARETDB1

با توجه به نتایجی که در جدول‌ها و شکل‌ها مشاهده می‌کنید متوجه می‌شویم که هر چند حساسیت روش پیشنهادی ما در تعدادی از مقادیر مثبت کاذب نسبت به موارد مشابه در دیگر مقاله‌ها کمتر شده است اما در کل معیار Fscore آن نسبت به بقیه بهتر شده است و این نشان می‌دهد که ترکیب ویژگی‌های ژرف (استخراج شده توسط انسان) و ویژگی‌های معنادار (استخراج شده توسط انسان) باعث می‌شود که سامانه مزیت هر دو دسته ویژگی را با هم داشته باشد و در به دست آوردن نتایج بهتر کمک کننده باشد.

