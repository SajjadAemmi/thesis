\chapter{نتایج و بحث}
\section{مقدمه}
در این فصل به تحلیل و بررسی نتایج آزمایشات انجام شده پرداخته می شود. ابتدا با استفاده از تحلیل ریاضی به مقایسه‌ی نتایج دبی محاسباتی توسط روابط ارائه شده در فصل دوم برای سرریزهای لبه‌تیز با دبی اندازه‌گیری شده در آزمایشگاه پرداخته می‌شود. در ادامه با استفاده از تحلیل ابعادی روابطی برای  ضریب دبی جریان عبوری از روی سرریز ساده‌ی مستطیلی و مرکب مستطیلی- مستطیلی با شکل لبه های متفاوت ارائه می‌شود.\\
در انتها نیز به بررسی اثر پارامترهای مختلف نظیر ارتفاع تاج سرریز(\lr{P}) و طول دهانه‌ی سرریز (\lr{L}) روی دبی و ضریب دبی جریان عبوری از روی سرریزهای ساده و مرکب مستطیلی پرداخته می‌شود.
\section{مقایسه‌ی دبی اندازه گیری شده با دبی محاسباتی}
در این قسمت به مقایسه‌ی دبی اندازه گیری شده در آزمایشگاه با دبی محاسباتی پرداخته می شود.
\subsection{سرریز لبه تیز ساده‌ی مستطیلی}
برای محاسبه‌ی دبی سرریزهای لبه‌تیز مستطیلی نسبتاً فشرده از روابط فرانسیس(\ref{eq2.12})، سوئیس(\ref{eq2.13})، کیندزواتر و کارتر(\ref{eq2.14}) و هندرسن(\ref{eq2.15}) استفاده شده و درصد خطای آنها نسبت به دبی اندازه‌گیری شده توسط رابطه‌ی زیر محاسبه شده است:
\begin{equation}\label{eq5.1}
\text{درصد خطا } =\Bigg|            \frac{\text{ دبی محاسباتی }  -     \text{         دبی اندازگیری شده}   }          {    \text{         دبی اندازگیری شده}    }
   \Bigg|  \times 100 = \Bigg|     \frac{Q_c - Q_m}{Q_m}     \Bigg|\times 100
\end{equation} 
در جدول (\ref{table5.1}) بازه‌ی دبی اندازه گیری شده و بار هیدرولیکی بالادست سرریز نسبت به تاج سرریز آورده شده است.  جدول (\ref{table5.2}) و نمودار (\ref{fig30}) نیز درصد خطای دبی محاسباتی و دبی اندازه‌گیری شده  با متغیر بودن طول سرریز (\lr{L}) و ثابت بودن ارتفاع تاج سرریز ($P=0.13m$) را برای 6 دبی مختلف و روش‌های متفاوت نشان می دهند.
\begin{table}
\centering
\caption{ بازه‌ی دبی اندازه گیری شده و بار هیدرولیکی بالادست سرریز در $P=0.13m$ با طول تاج های متفاوت}\label{table5.1}

\begin{tabular}{|c|c|c|c|}

\hline \multirow{2}{*}{ دبی، \lr{Q(m3/hr)} } & \multirow{2}{*}{\lr{ Hd(cm)}} &      \multirow{2}{*}{طول سرریز،\lr{L(cm)} }&\multirow{4}{*}{\rotatebox{90}{\mbox{سرریز مستطیلی لبه تیز ساده}}} \\  
 &   &  & \\ 
\cline{1-3}
 \multirow{3}{*}{21.1     $\sim$    6.08} & \multirow{3}{*}{ 10.89 $\sim$    4.57 }&  \multirow{3}{*}{11} &  \\ 
    &  &  &  \\  
    &  &  &  \\ \cline{1-3}
\multirow{3}{*}{33.9  $\sim$  6.5}  & \multirow{3}{*}{ 10.96      $\sim$        3.6}&  \multirow{3}{*}{ 16}&  \\  
   &  & & \\ 
    &  &  &  \\  
\hline 
\end{tabular} 
\end{table}


\begin{table}
\centering
\caption{ درصد خطای دبی محاسباتی و اندازه گیری شده با روشهای مختلف برای $P=0.13m$ با متغیر بودن طول تاج  } \label{table5.2}
\begin{tabular}{ |c|c|c|c|c| } 
 \hline
سرریز مستطیلی لبه تیز ساده      &  \multicolumn{4}{|c|}{    درصد خطای دبی محاسباتی و اندازه گیری شده     }  \\ \hline
          طول سرریز،\lr{L(cm)}                      &           \lr{Kindsvater} &    \lr{Swiss} & \lr{Francis}&      \lr{Henderson} \\ \hline
11   &        20.134  &	22.074   &	9.973    &	11.183\\  \hline
16  &    12.802  &   14.533  &    7.426     &     5.927  \\  \hline
\end{tabular}
\end{table}

\begin{diagram}[h]
    \centering
 \includegraphics[width=.7\linewidth]{38}
  \caption{مقایسه‌ی درصد خطای دبی محاسباتی و اندازه گیری شده در روش‌های مختلف با $P=0.13m$ و طول‌های متفاوت تاج سرریز}
  \label{fig30}
  \end{diagram}
همانطور که در نمودار(\ref{fig30}) دیده می‌شود، برای ($L=11cm$) روش فرانسیس کمترین خطا و روش سوئیس بیشترین خطا را دارد. برای ($L=16cm$) نیز روش هندرسن کمترین خطا و روش سوئیس بیشترین خطا را دارد.\\
در جدول (\ref{table5.3}) بازه‌ی دبی اندازه گیری شده و بار هیدرولیکی بالادست سرریز نسبت به تاج سرریز آورده شده است.  جدول (\ref{table5.4}) و نمودار (\ref{fig31}) نیز درصد خطای دبی محاسباتی و دبی اندازه‌گیری شده  با متغیر بودن ارتفاع تاج سرریز(\lr{P}) و ثابت بودن طول سرریز ($L=0.16m$) را برای 6 دبی مختلف و روش‌های متفاوت نشان می دهند.

\begin{table}
\centering
\caption{ بازه‌ی دبی اندازه گیری شده و بار هیدرولیکی بالادست سرریز در $L=0.16m$ با متغیر بودن ارتفاع تاج سرریز}\label{table5.3}

\begin{tabular}{|c|c|c|c|}

\hline \multirow{2}{*}{ دبی،\lr{Q(m3/hr)} } & \multirow{2}{*}{ \lr{Hd(cm)}} &      \multirow{2}{*}{ارتفاع تاج سرریز،\lr{P(cm)} }&\multirow{4}{*}{\rotatebox{90}{\mbox{\,\,\,\,سرریز مستطیلی لبه تیز }}} \\  
 &   &  & \\ 
\cline{1-3}
 \multirow{3}{*}{$32.2$   $\sim$  $6.4$} & \multirow{3}{*}{ $10.74$ $\sim$   $ 3.5$}&  \multirow{3}{*}{8} &  \\ 
    &  &  &  \\  
    &  &  &  \\ \cline{1-3}
\multirow{3}{*}{$33.9$  $\sim$ $6.5$}  & \multirow{3}{*}{ $10.96$      $\sim$        $3.6$}&  \multirow{3}{*}{ 13}&  \\  
   &  & & \\ 
    &  &  &  \\  
\hline 
\end{tabular} 
\end{table}


\begin{table}[h]
\centering
\caption{ درصد خطای دبی محاسباتی و اندازه گیری شده با روشهای مختلف برای $L=0.16m$ با متغیر بودن ارتفاع تاج سرریز } \label{table5.4}
\begin{tabular}{ |c|c|c|c|c| } 
 \hline
سرریز مستطیلی لبه تیز ساده      &  \multicolumn{4}{|c|}{    درصد خطای دبی محاسباتی و اندازه گیری شده     }  \\ \hline
         ارتفاع تاج سرریز،\lr{P(cm)}                      &          \lr{ Kindsvater} &    \lr{Swiss} &  \lr{Francis} &      \lr{Henderson} \\ \hline
8   &      $15.067$  &	$15.846$   &	$13.559$    &	$7.077$\\  \hline
13&$12.802$  &    $14.533$  &   $7.426 $      &     $5.927$  \\  \hline
\end{tabular}
\end{table}

\begin{diagram}[h]
    \centering
 \includegraphics[width=.7\linewidth]{39}

  \caption{مقایسه‌ی درصد خطای دبی محاسباتی و اندازه گیری شده در روش‌های مختلف با $P=0.13m$ و طول‌های متفاوت تاج سرریز  }
  \label{fig31}
  \end{diagram}
همانطور که در نمودار (\ref{fig31}) دیده می شود، روش هندرسن کمترین خطا و روش سوئیس بیشترین مقدار خطا را دارد. با توجه به نتایج بدست آمده مشاهده می شود که روش های فوق جواب های دقیقی بدست نمی دهند.
در نمودار (\ref{fig32}) درصد خطای دبی اندازه گیری شده و محاسباتی با روش های مختلف برای ارتفاع و طول های متفاوت، نسبت به ارتفاع تاج سرریز رسم شده است.
\begin{diagram}[h]
    \centering
\includegraphics[width=.7\linewidth]{40}

  \caption{ مقایسه‌ی درصد خطای دبی محاسباتی و اندازه گیری شده با روش‌های مختلف برای ارتفاع و طول های متفاوت  }
  \label{fig32}
  \end{diagram}
همانطور که در نمودار(\ref{fig32}) مشاهده می شود درصد خطای دبی محاسباتی و اندازه گیری شده با روش های مختلف با افزایش طول سرریز کاهش می‌یابد. در روش هندرسن با افزایش ارتفاع تاج سرریز درصد خطای دبی محاسباتی و انداز‌گیری شده تغییر ناچیزی کرده است. روش هندرسن کمترین درصد خطا و روش سوئیس و کیندزواتر بیشترین خطا را دارند.
\subsection{سرریز لبه تیز مرکب مستطیلی-مستطیلی}
برای محاسبه‌ی دبی در این حالت از ترکیب خطی روابط کیندزواتر، سوئیس، هندرسن و فرانسیس برای سرریز لبه تیز ساده‌ی مستطیلی استفاده شده و درصد خطای آن ها نسبت به دبی اندازه گیری شده بدست آمده و با یکدیگر مقایسه شده اند. 
در جدول (\ref{table5.5}) بازه‌ی دبی اندازه گیری شده و بار هیدرولیکی بالادست سرریز نسبت به تاج سرریز آورده شده است.  جدول (\ref{table5.6}) و نمودار (\ref{fig33}) نیز درصد خطای دبی محاسباتی و دبی اندازه‌گیری شده  با متغیر بودن طول سرریز ($L$) و ثابت بودن ارتفاع تاج سرریز ($P=0.13m$) را برای 6 دبی مختلف با روش‌های متفاوت نشان می دهد.

\begin{table}
\centering
\caption{  بازه‌ی دبی اندازه گیری شده و بار هیدرولیکی بالادست سرریز نسبت به تاج سرریز لبه تیز مرکب مستطیلی-مستطیلی در $p=0.13m$ با متغیر بودن طول تاج سرریز}\label{table5.5}

\begin{tabular}{|c|c|c|c|}

\hline \multirow{2}{*}{ دبی اندازه گیری شده\lr{Q(m3/hr)} } & \multirow{2}{*}{ $Hd(cm)$} &      \multirow{2}{*}{طول سرریز $(cm)$ }&\multirow{4}{*}{\rotatebox{90}{\mbox{سرریزلبه تیز
مرکب مستطیلی
}}} \\  
 &   &  & \\ 
\cline{1-3}
 \multirow{3}{*}{$44.56$   $\sim$  $20.54$} & \multirow{3}{*}{ $14.74$ $\sim$   $ 11.25$}&  \multirow{3}{*}{11} &  \\ 
    &  &  &  \\  
    &  &  &  \\ \cline{1-3}
\multirow{3}{*}{$52.5$  $\sim$ $34.6$}  & \multirow{3}{*}{ $13.9$     $\sim$        $11.1$}&  \multirow{3}{*}{ 16}&  \\  
   &  & & \\ 
    &  &  &  \\  
\hline 
\end{tabular} 
\end{table}

\begin{table}[h]
\centering
\caption{ درصد خطای دبی محاسباتی و اندازه گیری شده با روشهای مختلف برای P=0.13m با متغیر بودن طول تاج سرریز لبه تیز مرکب مستطیلی-مستطیلی } \label{table5.6}
\begin{tabular}{ |c|c|c|c|c| } 
 \hline
سرریز مستطیلی لبه تیز مرکب     &  \multicolumn{4}{|c|}{   درصد خطای دبی محاسباتی و اندازه گیری شده     }  \\ \hline
         طول سرریز \lr{ (cm)     }              &           \lr{Kindsvater} &   \lr{ Francis} & \lr{Swiss}  &     \lr{ Henderson} \\ \hline
11  &     $4.63$	 &  $9.392$	 & $ 4.83$    &	  $10.406$\\  \hline
16&       $6.36$	 & $10.968$  &	$6.5$   &	$12.695$\\  \hline
\end{tabular}
\end{table}


\begin{diagram}[h]
 \centering
 \includegraphics[width=.7\linewidth]{41}

  \caption{     مقایسه‌ی درصد خطای دبی محاسباتی و اندازه گیری شده به روش‌های مختلف با طول‌های متفاوت و $P=0.13m$  }
  \label{fig33}
  \end{diagram}
همانطور که در نمودار (\ref{fig33})  دیده می‌شود، در سرریز لبه‌تیز مرکب مستطیلی- مستطیلی با طول‌های مختلف و $P=0.13$ روش کیندزواتر و سوئیس کمترین درصد خطا و روش هندرسن و فرانسیس بیشترین خطا را دارند. \\
جدول(\ref{table5.7}) محدوده‌ی دبی اندازه گیری شده و بار هیدرولیکی بالادست سرریز نسبت به تاج سرریز را برای سرریز لبه تیز مرکب مستطیلی- مستطیلی با $L=0.16m$ و ارتفاع های متفاوت نشان می دهد. جدول(\ref{table5.8})  و نمودار (\ref{fig34}) نیز درصد خطای دبی اندازه گیری شده و محاسباتی توسط روش‌های مختلف را برای سرریز مذکور نشان می دهند.


\begin{table}
\centering
\caption{    بازه‌ی دبی اندازه گیری شده و بار هیدرولیکی بالادست سرریز نسبت به تاج سرریز لبه تیز مرکب مستطیلی-مستطیلی در $L=0.16m$ با متغیر بودن ارتفاع تاج سرریز}\label{table5.7}

\begin{tabular}{|c|c|c|c|}

\hline \multirow{2}{*}{ دبی اندازه گیری شده\lr{Q(m3/hr)} } & \multirow{2}{*}{\lr{ Hd(cm)}} &      \multirow{2}{*}{ارتفاع سرریز \lr{(cm)} }&\multirow{4}{*}{\rotatebox{90}{\mbox{سرریزلبه تیزمرکب مستطیلی
}}} \\  
 &   &  & \\ 
\cline{1-3}
 \multirow{3}{*}{$57.5$   $\sim$ $ 34.6$} & \multirow{3}{*}{ $14.8$ $\sim$   $ 11.3$}&  \multirow{3}{*}{8} &  \\ 
    &  &  &  \\  
    &  &  &  \\ \cline{1-3}
\multirow{3}{*}{$52.5$  $\sim$ $34.6$}  & \multirow{3}{*}{ $13.9$     $\sim$       $ 11.1$}&  \multirow{3}{*}{ 13}&  \\  
   &  & & \\ 
    &  &  &  \\  
\hline 
\end{tabular} 
\end{table}

\begin{table}[h]
\centering
\caption{        درصد خطای دبی محاسباتی و اندازه گیری شده با روشهای مختلف برای $L=0.16m$ با متغیر بودن ارتفاع تاج سرریز برای سرریز لبه تیز مرکب مستطیلی-مستطیلی } \label{table5.8}
\begin{tabular}{ |c|c|c|c|c| } 
 \hline
سرریز مستطیلی لبه تیز مرکب     &  \multicolumn{4}{|c|}{   درصد خطای دبی محاسباتی و اندازه گیری شده     }  \\ \hline
         ارتفاع سرریز (\lr{cm})                 &          \lr{ Kindsvater} &   \lr{ Francis} & \lr{Swiss}  &      \lr{Henderson} \\ \hline
8  &     $ 6.43$ &	$6.48$	&$4.66$	&$6.58$   \\  \hline
13&      $6.36$	&$10.968$	&$6.5$&	$12.695$\\  \hline
\end{tabular}
\end{table}

\begin{diagram}[h]
\centering
 \includegraphics[width=.7\linewidth]{42}
  \caption{       مقایسه‌ی درصد خطای دبی محاسباتی و اندازه گیری شده به روش‌های مختلف با ارتفاع های متفاوت و $L=0.16m$    }
  \label{fig34}
\end{diagram}
\noindent
برای $P=0.08m$ روش هندرسن خطای کمتری نسبت به سه روش دیگر دارد، برای $P=0.13m$ روش سوئیس و کیندزواتر کمترین خطا و روش هندرسن و فرانسیس بیشترین خطا را دارند. همانطور که مشاهده می‌شود درصد خطای دبی محاسباتی و اندازه‌گیری شده با دو روش سوئیس و کیندزواتر با افزایش ارتفاع تاج سرریز تغییر بسیار ناچیزی کرده است.\\
در نمودار (\ref{fig35}) درصد خطای دبی اندازه گیری شده و محاسباتی با روش های مختلف، برای ارتفاع و طول های متفاوت نسبت به ارتفاع تاج سرریز رسم شده است.
\begin{diagram}[h]
\centering
 \includegraphics[width=.7\linewidth]{43}

  \caption{       مقایسه‌ی درصد خطای دبی محاسباتی و اندازه گیری شده با روش‌های مختلف با ارتفاع و طول های متفاوت   }
  \label{fig35}
\end{diagram}
همانطور که مشاهده می‌شود، درصد خطای دبی اندازه‌گیری شده و محاسباتی با دو روش سوئیس و کیندزواتر با افزایش ارتفاع تاج سرریز کاهش و با افزایش طول سرریز افزایش یافته است. در دو روش هندرسن و فرانسیس با افزایش طول و ارتفاع تاج سرریز درصد خطا نیز افزایش یافته است.
\section{معادله‌ی ضریب دبی عبوری از روی سرریز}
در این بخش با استفاده از تحلیل ابعادی و نرم افزار اکسل روابطی برای تخمین ضریب دبی سرریز لبه تیز ساده‌ و لبه‌تیز مرکب مستطیلی، سرریز ساده و مرکب مستطیلی با لبه‌ی ربع دایره و سرریز ساده و مرکب مستطیلی با لبه‌ی نیم دایره ارائه می شود.
\subsection{معادله‌ی ضریب دبی سرریز لبه تیز ساده‌ی مستطیلی}\label{sub1.3}
هدف از این پژوهش استخراج یک رابطه بصورت بدون بعد برای تخمین ضریب دبی سرریزهای مستطیلی می باشد. فرم کلی ضریب دبی برای سرریزهای لبه تیز ساده‌ی مستطیلی با استفاده از رابطه‌ی (\ref{eq.21}) به شکل زیر می باشد:
\begin{equation}\label{eq5.2}
C_d=a(\frac{H_d}{B})^2+b(\frac{H_d}{B}) + c(\frac{H_d}{P})^2+ d(\frac{H_d}{P})+e(\frac{H_d}{L})^2+f(\frac{H_d}{L})+g
\end{equation} 
که در آن ضرایب  \lr{a} تا \lr{g} ضرایب ثابتی هستند که با استفاده از داده های آزمایشگاهی این پژوهش و با حداقل سازی مجموع مربعات خطا بدست می آیند. با توجه به معلوم بودن ضریب دبی و پارامترهای بدون بعد، ضرایب ثابت طوری محاسبه می شوند که ضریب دبی محاسباتی در مقایسه با ضریب دبی واقعی کمترین اختلاف را داشته باشد. به همین منظور از پارامتر آماری مجموع مربعات خطا به عنوان تابع هدف و از ضرایب مجهول \lr{a} تا  \lr{g} به عنوان متغیرهای تصمیم گیری استفاده شده و مراحل بهینه سازی ضرایب در محیط اکسل انجام شد.\\
در نهایت رابطه‌ی زیر برای برآورد ضریب دبی سرریزهای لبه تیز ساده‌ی مستطیلی بدست آمد:
\begin{equation}\label{eq5.3}
C_d=0.055 (\frac{H_d}{B})^2+0.49(\frac{H_d}{B})-0.0069(\frac{H_d}{P})^2-0.0282(\frac{H_d}{P})+0.072(\frac{H_d}{L})^2 -0.31(\frac{H_d}{L})+0.581
\end{equation}
بعد از استخراج رابطه‌ی مناسب برای تخمین ضریب دبی، دبی جریان عبوری با استفاده از رابطه‌ی \\ $Q=\frac{2}{3} C_d \sqrt{2g} LH_d^{1.5}$  ، با دقت بسیار بالا $ R^2=0.9996$ بدست می‌آید.
در جدول (\ref{table5.9}) درصد خطای دبی محاسباتی و اندازه‌گیری شده با متغیر بودن طول سرریز و ثابت بودن ارتفاع تاج سرریز ($P=0.13m$) برای 6 دبی مختلف نشان داده شده است.\\
\begin{table}[h]
\centering
\caption{     درصد خطای دبی محاسباتی و اندازه گیری شده با متغیر بودن طول سرریز و $ P=0.13m$     } \label{table5.9}
\begin{tabular}{ |P{3cm}|P{3cm}| } 
\hline
$L(cm)$ &     درصد خطا \\ \hline
11 & $0.82$ \\ \hline
16 & $0.794$ \\ \hline
\end{tabular}
\end{table}
در جدول(\ref{table5.10}) درصد خطای دبی محاسباتی و اندازه گیری شده با متغیر بودن ارتفاع تاج سرریز و ثابت بودن طول سرریز $L=0.16m$) ) برای 6 دبی آورده شده است.

\begin{table}[h]
\centering
\caption{     درصد خطای دبی محاسباتی و اندازه گیری شده با متغیر بودن ارتفاع تاج سرریز و $L=0.16m$   } \label{table5.10}
\begin{tabular}{ |P{3cm}|P{3cm}| } 
\hline
$P$ &     درصد خطا \\ \hline
8 & $0.863$ \\ \hline
13 & $0.794$ \\ \hline
\end{tabular}
\end{table}
\subsection{معادله‌ی ضریب دبی سرریز لبه تیز مرکب مستطیلی- مستطیلی}
فرم کلی ضریب دبی برای سرریزهای لبه تیز مرکب مستطیلی- مستطیلی با استفاده از رابطه‌ی (\ref{eq.29}) به شکل زیر می باشد:


\begin{equation}\label{eq5.3}
C_d=a(\frac{H_d}{L_1} )^2+b(\frac{H_d}{L_1} )+c(\frac{H_d}{P_1} )^2+d(\frac{H_d}{P_1})+e(\frac{H_d}{L_2} )^2+f(\frac{H_d}{L_2} )+g(\frac{H_d}{P_2} )^2+h(\frac{H_d}{P_2} )+i
\end{equation}
همانند بخش (\ref{sub1.3}) ضرایب \lr{a} تا \lr{i} با استفاده از داده‌های آزمایشگاهی و با حداقل‌سازی مجموع مربعات خطا بدست می‌آیند. نهایتاً رابطه‌ی ضریب دبی برای سرریزهای لبه‌تیز مرکب مستطیلی-‌ مستطیلی به فرم زیر بدست آمد: 
\begin{equation} \label{eq5.4}
\begin{split}
C_d=&0.481(\frac{H_d}{L_1} )^2 \pm 0.226(\frac{H_d}{L_1} )-0.012(\frac{H_d}{P_1} )^2 1.511(\frac{H_d}{P_1} )+0.863(\frac{H_d}{L_2} )^2 \\
& -0.154(\frac{H_d}{L_2} )+0.011(\frac{H_d}{P_2} )^2-6.666(\frac{H_d}{P_2} )+2.084
\end{split}
\end{equation}
 بعد از استخراج رابطه‌ی مناسب برای تعیین ضریب دبی، دبی جریان با استفاده از رابطه‌ی \\ $Q=(\frac{2}{3})C_d \sqrt{2g} H_d^{\frac{3}{2}} [L+L_2]$، با دقت بسیار بالا $R^2=0.9996$ بدست آمد.\\
در جدول (\ref{table5.11}) درصد خطای دبی محاسباتی و دبی اندازه‌گیری شده برای سرریز لبه تیز مرکب مستطیلی- مستطیلی با متغیر بودن طول سرریز و ثابت بودن ارتفاع تاج سرریز ($P=0.13m$) برای 6 دبی مختلف نشان داده شده است. \\
\begin{table}[h]
\centering
\caption{      درصد خطای دبی محاسباتی و اندازه گیری شده برای سرریز لبه تیز مرکب مستطیلی با متغیر بودن طول سرریز و $P=0.13m$   } \label{table5.11}
\begin{tabular}{ |P{3cm}|P{3cm}| } 
\hline
$L(cm)$ &     درصد خطا \\ \hline
11 & $0.218$ \\ \hline
13 & $0.424$ \\ \hline
\end{tabular}
\end{table}
در جدول(\ref{table5.12}) درصد خطای دبی محاسباتی و اندازه گیری شده با متغیر بودن ارتفاع تاج سرریز و ثابت بودن طول سرریز ($L=0.16m$)  برای 6 دبی آورده شده است.
\begin{table}[h]
\centering
\caption{       درصد خطای دبی محاسباتی و اندازه گیری شده با متغیر بودن ارتفاع تاج سرریز و $L=0.16m$  } \label{table5.12}
\begin{tabular}{ |P{3cm}|P{3cm}| } 
\hline
$P(cm)$ &     درصد خطا \\ \hline
8 & $0.984$ \\ \hline
13 & $0.414$ \\ \hline
\end{tabular}
\end{table}
\subsection{معادله‌ی ضریب دبی سرریز مستطیلی با لبه‌ی ربع دایره}
فرم کلی ضریب دبی برای سرریزهای ساده‌ی مستطیلی با لبه‌ی ربع دایره همانند بخش(\ref{sub1.3}) می باشد، که بعد از بهینه سازی ضرایب در محیط اکسل رابطه‌ی مناسب برای تخمین ضریب دبی به صورت زیر بدست می‌آید:
\begin{equation} \label{eq5.6}
\begin{split}
C_d=&-0.299(\frac{H_d}{B})^2+0.56(\frac{H_d}{B})+0.0566(\frac{H_d}{P})^2 \\
& -0.206(\frac{H_d}{P})+0.028(\frac{H_d}{L})^2-0.09(\frac{H_d}{L})+0.625
\end{split}
\end{equation}
بعد از تخمین ضریب دبی، دبی جریان عبوری از سرریز ساده‌ی مستطیلی با لبه‌ی ربع دایره با استفاده از رابطه‌ی  $Q=\frac{2}{3} C_d \sqrt{2g} LH_d^{1.5}$ با دقت $R^2=0.9997$ بدست آمده است. در جدول (\ref{table5.13}) درصد خطای دبی محاسباتی و دبی اندازه‌گیری شده برای سرریز مستطیلی با لبه‌ی ربع دایره با متغیر بودن طول سرریز و ثابت بودن ارتفاع تاج سرریز ($P=0.13m$) برای 7 دبی مختلف نشان داده شده است. 
\begin{table}[h]
\centering
\caption{       درصد خطای دبی محاسباتی و اندازه گیری شده برای سرریز مستطیلی با لبه ربع دایره با متغیر بودن طول سرریز و $P=0.13m$   } \label{table5.13}
\begin{tabular}{ |P{3cm}|P{3cm}| } 
\hline
$L(cm)$ &     درصد خطا \\ \hline
11 & $0.599$ \\ \hline
16 & $0.566$ \\ \hline
\end{tabular}
\end{table}
در جدول(\ref{table5.14}) درصد خطای دبی محاسباتی و اندازه گیری شده با متغیر بودن ارتفاع تاج سرریز و ثابت بودن طول سرریز ($L=0.16m$) برای 6 دبی آورده شده است.
\begin{table}[h]
\centering
\caption{       درصد خطای دبی محاسباتی و اندازه گیری شده برای سرریز مستطیلی با لبه‌ی ربع دایره با متغیر بودن ارتفاع تاج سرریز و $L=0.16m$   } \label{table5.14}
\begin{tabular}{ |P{3cm}|P{3cm}| } 
\hline
$P(cm)$ &     درصد خطا \\ \hline
8 & $0.604$ \\ \hline
13 & $0.566$ \\ \hline
\end{tabular}
\end{table}
\subsection{معادله‌ی ضریب دبی سرریز مرکب مستطیلی- مستطیلی با لبه‌ی ربع دایره}
ضریب دبی این نوع سرریز نیز همانند آنچه در بخش‌های قبل گفته شد با استفاده از رابطه‌ی زیر بدست می‌آید:
\begin{equation} \label{eq5.7}
\begin{split}
C_d=& 0.0657(\frac{H_d}{L_1} )^2+0.093(\frac{H_d}{L_1} )-0.06(\frac{H_d}{P_1} )^2+0.564(\frac{H_d}{P_1} )\\
& +0.237(\frac{H_d}{L_2} )^2-0.196(\frac{H_d}{L_2} )+0.449(\frac{H_d}{P_2} )^2-2.719(\frac{H_d}{P_2} )+1.437
\end{split}
\end{equation}
دبی این سرریزها نیز با دقت بسیار بالا $R^2=0.998$  در محیط اکسل محاسبه شده است. جدول(\ref{table5.15}) درصد خطای دبی محاسباتی و اندازه گیری شده برای سرریز مرکب مستطیلی- مستطیلی با متغیر بودن طول سرریز و $P=0.13m$ و جدول(\ref{table5.16}) نیز درصد خطا را برای این نوع سرریز با متغیر بودن ارتفاع تاج سرریز و $L=0.16m$  نشان می دهد.
\begin{table}[h]
\centering
\caption{       درصد خطای دبی محاسباتی و اندازه گیری شده برای سرریز مرکب مستطیلی- مستطیلی با لبه‌ی ربع دایره با متغیر بودن طول سرریز و $P=0.13m$  } \label{table5.15}
\begin{tabular}{ |P{3cm}|P{3cm}| } 
\hline
$L(cm)$ &     درصد خطا \\ \hline
11 & $0.78$ \\ \hline
16 & $0.811$ \\ \hline
\end{tabular}
\end{table}

\begin{table}[h]
\centering
\caption{    درصد خطای دبی محاسباتی و اندازه گیری شده برای سرریز مرکب مستطیلی- مستطیلی با لبه‌ی ربع دایره با متغیر بودن ارتفاع تاج سرریز و $L=0.16m$   } \label{table5.16}
\begin{tabular}{ |P{3cm}|P{3cm}| } 
\hline
$P(cm)$ &     درصد خطا \\ \hline
8 & $0.888$ \\ \hline
13 & $0.811$ \\ \hline
\end{tabular}
\end{table}
\subsection{معادله‌ی ضریب دبی سرریز ساده‌ی مستطیلی با لبه‌ی نیم دایره}
ضریب دبی برای سرریزهای مستطیلی با لبه‌ی نیم دایره همانند بخش(\ref{sub1.3}) مطابق رابطه‌ی زیر بدست می آید: 
\begin{equation} \label{eq5.8}
\begin{split}
C_d=&0.117(\frac{H_d}{P})^2-0.349(\frac{H_d}{P})+0.341(\frac{H_d}{B})^2-0.411(\frac{H_d}{B})\\
& -0.059(\frac{H_d}{L})^2+0.249(\frac{H_d}{L})+0.796
\end{split}
\end{equation}

دبی این نوع سرریز نیز با استفاده از رابطه‌ی $Q=\frac{2}{3} C_d \sqrt{2g} LH_d^{1.5}$،     با دقت بسیار بالا $R^2=0.9994$  بدست آمد. جدول(\ref{table5.17}) درصد خطای دبی محاسباتی و اندازه گیری شده برای سرریز ساده‌ی مستطیلی با لبه‌ی نیم دایره با متغیر بودن طول سرریز و $P=0.13m$ و جدول(\ref{table5.18}) درصد خطا را برای این نوع سرریز با متغیر بودن ارتفاع تاج سرریز و $ L=0.16m $ نشان می دهد.

\begin{table}[h]
\centering
\caption{    درصد خطای دبی محاسباتی و اندازه گیری شده برای سرریز مستطیلی با لبه نیم دایره با متغیر بودن طول سرریز و $P=0.13m$  } \label{table5.17}
\begin{tabular}{ |P{3cm}|P{3cm}| } 
\hline
$L(cm)$ &     درصد خطا \\ \hline
11 & $0.837$ \\ \hline
16 & $0.641$ \\ \hline
\end{tabular}
\end{table}

\begin{table}[h]
\centering
\caption{    درصد خطای دبی محاسباتی و اندازه گیری شده برای سرریز مستطیلی با لبه‌ی نیم دایره با متغیر بودن ارتفاع تاج سرریز و $L=0.16m$   } \label{table5.18}
\begin{tabular}{ |P{3cm}|P{3cm}| } 
\hline
$P(cm)$ &     درصد خطا \\ \hline
8 & $0.638$ \\ \hline
13 & $0.641$ \\ \hline
\end{tabular}
\end{table}

\subsection{ معادله‌ی ضریب دبی سرریز مرکب مستطیلی- مستطیلی با لبه‌ی نیم دایره}
همانطور که در بخش (\ref{sub1.3}) آمده است با استفاده از نرم افزار اکسل رابطه‌ی ضریب دبی بصورت زیر بدست می آید:
\begin{equation} \label{eq5.9}
\begin{split}
C_d=&0.4(\frac{H_d}{L_1} )^2+0.486(\frac{H_d}{L_1} )+0.007(\frac{H_d}{P_1} )^2+2.407(\frac{H_d}{P_1} )\\
& +2.321(\frac{H_d}{L_2 })^2-1.538(\frac{H_d}{L_2} )+0.338(\frac{H_d}{P_2} )^2-11.43(\frac{H_d}{P_2} )+3.187
\end{split}
\end{equation}
دبی جریان عبوری از این سرریزها نیز با دقت بسیار بالا $R^2=0.998$  بدست آمد. جدول (\ref{table5.19}) درصد خطای دبی محاسباتی و اندازه گیری شده برای سرریز مرکب مستطیلی- مستطیلی با لبه‌ی نیم دایره با متغیر بودن طول سرریز و $P=0.13m$ و جدول (\ref{table5.20}) نیز درصد خطا را برای این سرریز با متغیر بودن ارتفاع تاج سرریز و $  L=0.16m $ نشان می دهد.
\begin{table}[h]
\centering
\caption{    درصد خطای دبی محاسباتی و اندازه گیری شده برای سرریز مرکب مستطیلی- مستطیلی با لبه‌ی نیم دایره با متغیر بودن طول سرریز و $P=0.13m$       } \label{table5.19}
\begin{tabular}{ |P{3cm}|P{3cm}| } 
\hline
$L(cm)$ &     درصد خطا \\ \hline
11 & $0.746$ \\ \hline
16 & $0.622$ \\ \hline
\end{tabular}
\end{table}


\begin{table}[h]
\centering
\caption{           درصد خطای دبی محاسباتی و اندازه گیری شده برای سرریز مرکب مستطیلی- مستطیلی با لبه‌ی نیم دایره با متغیر بودن ارتفاع تاج سرریز و $L=0.16m$    } \label{table5.20}
\begin{tabular}{ |P{3cm}|P{3cm}| } 
\hline
$P(cm)$ &     درصد خطا \\ \hline
8 & $0.942$ \\ \hline
13 & $0.622$ \\ \hline
\end{tabular}
\end{table}
\section{اثر ارتفاع تاج سرریز روی ضریب دبی}
در این قسمت به بررسی اثر ارتفاع تاج سرریز روی ضریب دبی سرریزهای ساده و مرکب مستطیلی با شکل‌های مختلف لبه می پردازیم.
\begin{diagram}[h]
\centering
 \includegraphics[width=.7\linewidth]{44}

  \caption{      مقایسه‌ی ارتفاع‌های متفاوت تاج سرریز در سرریز ساده‌ی لبه‌تیز مستطیلی و $L=0.16m$   }
  \label{fig36}
\end{diagram}
همانطور که در نمودار(\ref{fig36}) مشاهده می‌شود، در سرریزهای ساده‌ی مستطیلی لبه‌تیز در یک ارتفاع تاج سرریز با افزایش مقدار $H_d/P$ ضریب دبی کاهش می‌یابد. همینطور در یک مقدار مشخص $H_d/P$ نیز با افزایش ارتفاع تاج سرریز ضریب دبی کاهش می‌یابد.
\begin{diagram}[h]
\centering
 \includegraphics[width=.7\linewidth]{45}

  \caption{      مقایسه‌ی ارتفاع‌های متفاوت تاج سرریز در سرریز لبه‌تیز مرکب مستطیلی و $L=0.16m$   }
  \label{fig37}
\end{diagram}
همانطور که در نمودار فوق مشاهده می‌شود در سرریز لبه‌تیز مرکب مستطیلی در یک ارتفاع تاج ثابت، ضریب دبی با افزایش مقدار $H_d/P$ کاهش می‌یابد و در یک $H_d/P$ ثابت با افزایش ارتفاع تاج سرریز، ضریب دبی کاهش می‌یابد.
\begin{diagram}[h]
\centering
 \includegraphics[width=.7\linewidth]{46}

  \caption{     مقایسه‌ی ارتفاع‌های متفاوت تاج سرریز در سرریز ساده‌ی مستطیلی با لبه‌ی ربع دایره و $L=0.16m$   }
  \label{fig38}
\end{diagram}
همانطور که در نمودار(\ref{fig38}) مشاهده می‌شود، در سرریزهای ساده‌ی مستطیلی با لبه‌ی ربع دایره در یک ارتفاع تاج ثابت، ضریب دبی با افزایش $H_d/P$ کاهش می‌یابد و در یک مقدار ثابت $H_d/P$ ، با افزایش ارتفاع تاج سرریز ضریب دبی نیز افزایش می‌یابد.
\begin{diagram}[h]
\centering
 \includegraphics[width=.7\linewidth]{47}

  \caption{      مقایسه‌ی ارتفاع‌های متفاوت تاج سرریز در سرریز مرکب مستطیلی با لبه‌ی ربع دایره و $L=0.16m$   }
  \label{fig39}
\end{diagram}
همانطور که در نمودار فوق مشاهده می‌شود، در سرریزهای مرکب مستطیلی با لبه‌ی ربع دایره برای یک مقدار مشخص ارتفاع تاج سرریز با افزایش مقدار $H_d/P$ ضریب دبی کاهش می‌یابد، هم‌چنین در یک $H_d/P$ ثابت با افزایش ارتفاع تاج سرریز ضریب دبی کاهش می‌یابد.
\begin{diagram}[h]
\centering
 \includegraphics[width=.7\linewidth]{48}

  \caption{      مقایسه‌ی ارتفاع‌های متفاوت تاج سرریز در سرریز مرکب مستطیلی با لبه‌ی ربع دایره و $L=0.16m$   }
  \label{fig40}
\end{diagram}
همانگونه که در نمودار (\ref{fig40}) نشان داده شده است، ضریب دبی سرریزهای ساده‌ی مستطیلی با لبه‌ی نیم‌دایره در یک ارتفاع تاج مشخص با اقزایش مقدار   $H_d/P$   کاهش می‌یابد. هم‌چنین در یک مقدار مشخص  $H_d/P$ نیز با افزایش ارتفاع تاج سرریز ضریب دبی افزایش می‌یابد.
\begin{diagram}[h]
\centering
 \includegraphics[width=.7\linewidth]{49}
  \caption{   مقایسه‌ی ارتفاع‌های متفاوت تاج سرریز در سرریز مرکب مستطیلی با لبه‌ی نیم دایره و $L=0.16m$  }
  \label{fig41}
\end{diagram}
همانطور که در نمودار (\ref{fig41}) مشاهده می‌شود، در سرریزهای مرکب مستطیلی با لبه‌ی نیم دایره برای یک مقدار مشخص ارتفاع تاج سرریز با افزایش مقدار   $H_d/P$   ضریب دبی کاهش می‌یابد، هم‌چنین در یک   $H_d/P$   ثابت با افزایش ارتفاع تاج سرریز ضریب دبی کاهش می‌یابد.
\section{اثر طول دهانه‌ی سرریز روی ضریب دبی}
در این قسمت به بررسی اثر طول دهانه‌ی مرکزی سرریز روی ضریب دبی سرریزهای ساده و مرکب مستطیلی با شکل‌های مختلف لبه می‌پردازیم.
\begin{diagram}[h]
\centering
 \includegraphics[width=.7\linewidth]{50}
  \caption{    مقایسه‌ی ضریب دبی و $H_d/P$ در سرریز ساده‌ی لبه‌تیز مستطیلی با $P=0.13m$ و ابعاد مختلف دهانه‌ی سرریز  }
  \label{fig42}
\end{diagram}

\begin{diagram}[h]
\centering
 \includegraphics[width=.7\linewidth]{51}
  \caption{   مقایسه‌ی ضریب دبی و $H_d/P$ در سرریز لبه‌تیز مرکب  مستطیلی با $P=0.13m$ و ابعاد مختلف دهانه‌ی سرریز  }
  \label{fig43}
\end{diagram}
همانطور که در نمودارهای (\ref{fig42}) و (\ref{fig43}) مشاهده می‌شود، ضریب دبی سرریزهای ساده و مرکب مستطیلی لبه‌تیز در یک مقدار مشخص  $H_d/P$    با افزایش طول دهانه‌ی مرکزی سرریز افزایش می‌یابد و در یک مقدار ثابت $L$ با افزایش   $H_d/P$    ضریب دبی کاهش می‌یابد.
\begin{diagram}[h]
\centering
 \includegraphics[width=.7\linewidth]{52}
  \caption{  مقایسه‌ی ضریب دبی و     $H_d/P$  در سرریز ساده‌ی  مستطیلی با لبه‌ی ربع دایره با $P=0.13m$ و ابعاد مختلف دهانه‌ی سرریز  }
  \label{fig43}
\end{diagram}

\begin{diagram}[h]
\centering
 \includegraphics[width=.7\linewidth]{53}
  \caption{  مقایسه‌ی ضریب دبی و $H_d/P$ در سرریز مرکب  مستطیلی با لبه‌ی ربع دایره با $P=0.13m$ و ابعاد مختلف دهانه‌ی سرریز   }
  \label{fig44}
\end{diagram}
همانطور که در نمودارهای (\ref{fig43}) و (\ref{fig44}) مشاهده می‌شود، ضریب دبی سرریزهای ساده و مرکب مستطیلی با لبه‌ی ربع دایره در یک مقدار ثابت $L$ با افزایش  $H_d/P$ کاهش می‌یابد. طبق نمودار(\ref{fig43}) در یک سرریز ساده‌ی مستطیلی با لبه‌ی ربع دایره در یک مقدار ثابت $H_d/P$ با افزایش طول دهانه‌ی سرریز ضریب دبی نیز افزایش می‌یابد و در نمودار(\ref{fig44}) نیز مشاهده می‌شود که در یک سرریز مرکب مستطیلی با لبه‌ی ربع دایره در یک مقدار ثابت   $H_d/P$  با افزایش طول دهانه‌ی سرریز ضریب دبی کاهش می‌یابد.
\begin{diagram}[h]
\centering
 \includegraphics[width=.7\linewidth]{54}
  \caption{  مقایسه‌ی ضریب دبی و  $H_d/P$  در سرریز ساده‌ی  مستطیلی با لبه‌ی نیم‌دایره با $P=0.13m$ و ابعاد مختلف دهانه‌ی سرریز   }
  \label{fig45}
\end{diagram}

\begin{diagram}[h]
\centering
 \includegraphics[width=.7\linewidth]{55}
  \caption{    مقایسه‌ی ضریب دبی و  $H_d/P$  در سرریز مرکب  مستطیلی با لبه‌ی نیم‌دایره با $P=0.13m$ و ابعاد مختلف دهانه‌ی سرریز  }
  \label{fig46}
\end{diagram}
همانطور که در نمودارهای (\ref{fig45}) و (\ref{fig46}) مشاهده می‌شود در سرریزهای ساده و مرکب مستطیلی با لبه‌ی نیم دایره در یک مقدار ثابت       $H_d/P$     با افزایش طول دهانه‌ی سرریز ضریب دبی کاهش می‌یابد و در یک مقدار ثابت $L$ با افزایش  $H_d/P$  ضریب دبی کاهش می‌یابد.
\section{اثر شکل لبه‌ی سرریز روی ضریب دبی}
در این بخش به بررسی اثر شکل لبه‌ی سرریز روی ضریب دبی سرریزهای ساده و مرکب مستطیلی می‌پردازیم.
\begin{diagram}[h]
\centering
 \includegraphics[width=.7\linewidth]{56}
  \caption{     مقایسه‌ی ضریب دبی و  $H_d/P$      در سرریز ساده‌ی مستطیلی با شکل‌های مختلف لبه( $P=0.13m$ و $ L=0.16m$) }
  \label{fig47}
\end{diagram}


\begin{diagram}[h]
\centering
 \includegraphics[width=.7\linewidth]{57}
  \caption{     مقایسه‌ی ضریب دبی و $H_d/P$    در سرریز ساده‌ی مستطیلی با شکل‌های مختلف لبه ( $P=0.13m$ و $ L=0.11m $) }
  \label{fig48}
\end{diagram}
\noindent
طبق نمودارهای (\ref{fig47}) و (\ref{fig48}) در سرریزهای ساده‌ی مستطیلی با یک مقدار ثابت   $H_d/P$  ، سرریزهای مستطیلی لبه‌تیز کمترین ضریب دبی و سرریزهای مستطیلی با لبه‌ی نیم‌دایره بیشترین ضریب دبی را دارند.
\begin{diagram}[h]
\centering
 \includegraphics[width=.7\linewidth]{58}
  \caption{    مقایسه‌ی ضریب دبی و   $H_d/P$ در سرریزهای مرکب مستطیلی با شکل‌های مختلف لبه ( $P=0.13m$ و $L=0.16m$)}
  \label{fig49}
\end{diagram}

\begin{diagram}[h]
\centering
 \includegraphics[width=.7\linewidth]{59}
  \caption{  مقایسه‌ی ضریب دبی و    $H_d/P$ در سرریزهای مرکب مستطیلی با شکل‌های مختلف لبه( $P=0.08m$ و $L=0.16m$)   }
  \label{fig50}
\end{diagram}
\noindent
همانطور که در نمودارهای (\ref{fig49}) و (\ref{fig50}) مشاهده می‌شود در سرریزهای مرکب مستطیلی با $H_d/P$   ثابت، سرریزهای مستطیلی لبه‌تیز کمترین ضریب دبی و سرریزهای مستطیلی با لبه‌ی نیم‌دایره بیشترین ضریب دبی را دارند.
\section{اثر استغراق روی ضریب دبی}
در این قسمت به بررسی اثر استغراق بر روی ضریب دبی سرریزهای ساده و مرکب مستطیلی پرداخته می‌شود.
\begin{diagram}[h]
\centering
 \includegraphics[width=.7\linewidth]{60}
  \caption{  نمودار تغییرات ضریب دبی نسبی($\frac{C_{ds}}{C_d}$  ) با درصد استغراق ( $\frac{H_1}{H_d}$ ) برای سرریز ساده‌ی مستطیلی لبه‌تیز با $p=0.08m$  و $L=0.16m$  }
  \label{fig51}
\end{diagram}

\begin{diagram}[h]
\centering
 \includegraphics[width=.7\linewidth]{61}
  \caption{                -  نمودار تغییرات ضریب دبی نسبی ($\frac{C_{ds}}{C_d}$  ) با درصد استغراق ($\frac{ H_1}{H_d}$ ) برای سرریز ساده‌ی مستطیلی با لبه‌ی ربع دایره با $p=0.08m$ و $L=0.16m$      }
  \label{fig52}
\end{diagram}

\begin{diagram}[h]
\centering
 \includegraphics[width=.7\linewidth]{62}
  \caption{              نمودار تغییرات ضریب دبی نسبی ($\frac{C_{ds}}{C_d}$) با درصد استغراق ($\frac{H_1}{H_d}$) برای سرریز ساده‌ی مستطیلی با لبه‌ی نیم دایره با $p=0.08m$ و $  L=0.16m$      }
  \label{fig53}
\end{diagram}
همانطور که در نمودارهای (\ref{fig51})، (\ref{fig52}) و (\ref{fig53}) مشاهده می‌شود با افزایش درصد استغراق، ضریب دبی نسبی جریان(ضریب دبی جریان مستغرق نسبت به ضریب دبی جریان در حالت آزاد) کاهش می‌یابد، به عبارت دیگر با افزایش میزان استغراق ضریب دبی جریان کمتر می‌شود. ( $H_1$ = فاصله‌ی تراز آب پایین دست تا تاج سرریز، $C_ds$ = ضریب دبی جریان مستغرق و $C_d$ = ضریب دبی در حالت جریان آزاد می‌باشد.)
\begin{diagram}[h]
\centering
 \includegraphics[width=.7\linewidth]{63}
  \caption{              نمودار تغییرات ضریب دبی نسبی ($\frac{C_{ds}}{C_d}$) با درصد استغراق ($\frac{H_1}{H_d}$) برای سرریزهای مرکب  مستطیلی با شکل‌های مختلف لبه $p=0.08m$  و  $L=0.11m$      }
  \label{fig54}
\end{diagram}
با توجه به نمودار(\ref{fig54}) مشاهده می‌شود که با افزایش درصد استغراق، ضریب دبی نسبی جریان برای هر سه سرریز مرکب مستطیلی لبه‌تیز، مرکب مستطیلی با لبه‌ی نیم دایره و مرکب مستطیلی با لبه‌ی ربع دایره کاهش می‌یابد.














